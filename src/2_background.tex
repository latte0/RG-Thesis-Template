\chapter{背景}
\label{background}

本章では本研究の背景について述べる.
まず, 教師なし学習における生成モデルの立場を明確にする.
まず機械学習におけるニューラルネットワークの役割について明確にする。

そして現在, 研究が活発となっている深層ニューラルネットワークと深層
生成モデルを概説し, 深層ニューラルネットワークと深層生成モデルの抱える問題点を明
らかにする.

そして現在研究が活発になっている機械学習における活性化関数の動向について概説し、問題点を明確にする。




 深層ニューラルネットワークの問題を解決するための手法の1つであるガウ
ス過程回帰とその応用であるガウス過程潜在変数モデルを導入する.

最後に、実社会において機械学習を行う上での問題点や課題を述べ、本研究が取り組むべき課題を明確にする。



\section{ニューラルネットワーク}
・ラーニングレート
・初期値、
・レギュラライザー(l1ノルムなど)
・optimizer
この辺について
\section{活性化関数}
\section{統計学における位置付け}
\section{ノンパラメトリックモデル}
\section{カーネル法}
\section{実社会における学習の問題点}


\if0
\begin{figure}[h]
    \begin{center}
        \includegraphics[scale=0.4]{./img/hashrate.png}
        \caption{2017年1月のハッシュレート分布 出典:Blockchain.info\cite{bitcoinhashrate}}
        \label{img:hashrate}
    \end{center}
\end{figure}
\fi
