\chapter{背景}
\label{background}

本章では本研究の背景について述べる.
まず機械学習における活性化関数のの役割について明確にする。
活性化関数について概説し、現在の機械学習における活性化関数の抱える問題点を明らかにする。

次に、活性化関数の他に、ニューラルネットワークにおける精度を向上させるいくつかの構成要素について述べる。


また、統計学において活性化関数というのがどのような


次に、ノンパラメトリックモデルを

最後に、実社会において機械学習を行う上での問題点や課題を述べ、本研究が取り組むべき課題を明確にする。



\section{活性化関数}
\section{ニューラルネットワークの}
・ラーニングレート
・初期値、
・レギュラライザー(l1ノルムなど)
・optimizer
この辺について
\section{統計学における位置付け}
\section{ノンパラメトリックモデルとカーネル法}
\section{実社会における学習の問題点}


\if0
\begin{figure}[h]
    \begin{center}
        \includegraphics[scale=0.4]{./img/hashrate.png}
        \caption{2017年1月のハッシュレート分布 出典:Blockchain.info\cite{bitcoinhashrate}}
        \label{img:hashrate}
    \end{center}
\end{figure}
\fi
