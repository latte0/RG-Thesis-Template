\chapter{提案手法}
\label{proposed}

本章では提案手法について述べる.

\section{概要}

現在、深層学習に利用できる活性化関数が研究されている。最近では、中間層にReLUを用い、最終的な出力層にシグモイドを用いた組み合わせがよく用いられている。
しかし、これらの組み合わせは経験的なものだけでなく、データに対する人間の知識が事前に必要とされる。


また、本研究はSiluやmish、またxxxxxxからの単調増加性の仮定を外した点に着目した。ReLUやsigmoidといった関数はあくまでリンク関数の概念に相当していたこと考えると、
単調増加性というのは機械学習の観点では本来必要であったものかどうか問われるものであった。
統計学のSIMの法でも同様に単調増加性について仮定しているようであるが、ほんけんきゅうからはそれを外してやる。

そのため本研究ではSIMの考え方の中の単調増加性の過程を外し、

本研究では、統計学の世界で使われているSIMのノンパラメトリック法を用いて、活性化関数に我々のリンク関数推定法を適用しました。


そうすることで、より精度の高い結果を導き出すことができます。
関数の形式はカーネル関数であり、入力の出力を一つの式で表現することができる。
これにより、深層学習に利用できる程度に計算コストを削減することができる。
また、今回の実験では、活性化関数を既存の関数から選択するのではなく、状況に応じた関数、つまり活性化関数の形を作ることができる。
そうすることで、関数全体から逆算して最適な関数を見つけることができる。これにより、これまでディープラーニングで課題とされてきた活性化関数の選択の問題を解決することが可能となり、新たなアプローチが可能になるのではないでしょうか。


\section{ノンパラメトリック}


現状ではディープラーニングに活かせるようなノンパラメトリックに推定する活性化関数は研究されておらず、経験的に中間層ではRelu、最終的なアウトプット層ではデータセットに合わせてSigmoidが使われることが多い。
しかしながらこれらの組み合わせは経験的であるだけではなく、データに対する人知見が事前に必要である。
本研究では、統計の世界で使われていたSIMでのノンパラメトリックな手法を用いて行われていたリンク関数の推定方法を活性化関数に応用する。
そうすることにより、経験的な知見による活性化関数の選択という行為を行わずともより高い精度の結果を導けるのではないかということである。
関数の形式はカーネル関数を用いることで、入力に対しての出力を一つの式で表せるようにする。そうすることでディープラーニングでも使えるぐらいの少ない計算コストが実現できる。


また、この実験により状況に応じた適切か活性化関数の形を既存のものから選択するのではなく、関数全体の中から逆算できると考えている。それにより、ディープラーニングの課題であった、活性化関数の選択問題という課題も新しいアプローチで解決できると考えている。


\section{活性化関数}
本節では既存の活性化関数の問題点を具体的な事例を交えて考える。


\section{kernel活性関数}



本論文で私が提案する活性化関数を以下の数式で表現する。


\begin{eqnarray}
G(X_iw)=\frac{\sum_{i\neq j} K\left(\frac{X^{calc}_j w - X_i w}{h_{calc}}\right)Y^{calc}_j}{\sum_{i\neq j} K\left(\frac{X^{calc}_j w - X_i w}{h_{calc}}\right)}
\end{eqnarray}

ichimuraではデータセットの数だけで表現していたが、一部を省略することにより少ない変数で表現することに成功した。


\section{アルゴリズム}


\begin{algorithm}[!t]
	\caption{\KAF}
	\label{alg:fixed-u-alg}
\begin{algorithmic}
	\STATE {\bfseries Input:} data $\langle (x_i, y_i) \rangle_{i=1}^m \in
	\reals^d \times [0, 1]$, $u: \reals \rightarrow [0, 1]$.
	\STATE $w^1 := 0$;
	\FOR {$t = 1, 2, \ldots$}
	\STATE $h^t(x) := u(w^t \cdot x)$;
	\STATE $w^{t+1} := w^t + \displaystyle\fraconem \sumionetom (y_i - u(w^t
	\cdot x_i)) x_i$;
	\ENDFOR
\end{algorithmic}
\end{algorithm}





%%% Local Variables:
%%% mode: japanese-latex
%%% TeX-master: "../bthesis"
%%% End:
