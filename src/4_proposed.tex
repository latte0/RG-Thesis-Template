\chapter{提案手法}
\label{proposed}

本章では提案手法について述べる.

\section{概要}


\section{ノンパラメトリック}


現状ではディープラーニングに活かせるようなノンパラメトリックに推定する活性化関数は研究されておらず、経験的に中間層ではRelu、最終的なアウトプット層ではデータセットに合わせてSigmoidが使われることが多い。
しかしながらこれらの組み合わせは経験的であるだけではなく、データに対する人知見が事前に必要である。
本研究では、統計の世界で使われていたSIMでのノンパラメトリックな手法を用いて行われていたリンク関数の推定方法を活性化関数に応用する。
そうすることにより、経験的な知見による活性化関数の選択という行為を行わずともより高い精度の結果を導けるのではないかということである。
関数の形式はカーネル関数を用いることで、入力に対しての出力を一つの式で表せるようにする。そうすることでディープラーニングでも使えるぐらいの少ない計算コストが実現できる。
また、この実験により状況に応じた適切か活性化関数の形を既存のものから選択するのではなく、関数全体の中から逆算できると考えている。それにより、ディープラーニングの課題であった、活性化関数の選択問題という課題も新しいアプローチで解決できると考えている。


\section{活性化関数}
本節では既存の活性化関数の問題点を具体的な事例を交えて考える。

\section{kernel活性関数}

本論文で私が提案する活性化関数を


%%% Local Variables:
%%% mode: japanese-latex
%%% TeX-master: "../bthesis"
%%% End:
