\chapter{実装}
\label{implementation}

本章では提案手法の実装について述べる.

\section{概要}
他の活性化関数と適当に比較するために、以下の条件を比較して実験を行う。
・ラーニングレート
・初期値、
・レギュラライザー(l1ノルムなど)
・optimizer
・テストデータ
既存のものと比較している"


\section{実装手法}
\section{活性化関数}
\section{アルゴリズム}


\begin{algorithm}[!t]
	\caption{\GLMt}
	\label{alg:fixed-u-alg}
\begin{algorithmic}
	\STATE {\bfseries Input:} data $\langle (x_i, y_i) \rangle_{i=1}^m \in
	\reals^d \times [0, 1]$, $u: \reals \rightarrow [0, 1]$.
	\STATE $w^1 := 0$;
	\FOR {$t = 1, 2, \ldots$}
	\STATE $h^t(x) := u(w^t \cdot x)$;
	\STATE $w^{t+1} := w^t + \displaystyle\fraconem \sumionetom (y_i - u(w^t
	\cdot x_i)) x_i$;
	\ENDFOR
\end{algorithmic}
\end{algorithm}


\section{}
\section{}

%%% Local Variables:
%%% mode: japanese-latex
%%% TeX-master: "../bthesis"
%%% End:
