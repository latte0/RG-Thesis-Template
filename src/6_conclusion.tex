\chapter{結論}
\label{conclusion}

本章では実験の結果に対する考察を行い、第1章\ref{introductionで述べた貢献\ref{kouken}にどのように繋がったか考える。

そして、提案手法の利点と限界について述べ、今後の課題及び方針を示す。


\section{本研究のまとめ}

カーネルを使った汎用的な関数でニューラルネットの最終層を置き換えることで、実際に精度の向上を図ることができた。

この実験を通して、ReLUやシグモイドと同等かそれ以上の結果が得られていることがわか流。
重要な結果は、データセットの形状がわからなくても、K-AFの形状がSigmoidに近いことである。 
また、決定木の実験によく使われるワインデータセットでは、式による単純な分類が難しいとされているが、Sigmoidなどよりも良い結果が得られている。
また、シグモイドは学習の仕方によっては特異点にはまってしまうこともありますが カーネルはこれを回避することができた。
これらの結果により、ブラックボックス化された活性化関数選択問題の解決に近づいたことがわかる。




\section{本研究の課題}
スカラー値が大きなデータセットにおいては、その推論の精度が低下するだけではなく、勾配が消失して計算の継続が難しくなることがある。
これらを解消するために、適切なニューラルネットワークの構成をより一層研究するだけではなく、それらが起こる原因を探求する必要がある。



\section{将来的な展望}


本研究ではカーネル密度推定を用いて機械学習における学習精度の向上を目指した。
提案手法が幅広いデータセットにおいて有益な結果をしますことを実験により明らかにし、それが実用的なデータでも応用可能であることを示した。
活性化関数を汎用的に推論するという論文は未だ少なく研究分野として今後非常に注目すべきであると考えている。 
自動運転などの産業分野においても有用なモデルへの応用、
また、K-AFが汎用的な活性化関数の代表として初学者や非エンジニアが扱いやすい道具として応用されることを望んでいる。
本研究における提案手法をより効率的で使いやすいものにすることで深層学習に関する諸研究へと応用され、、 機械学習の応用分野に対してさらなる貢献ができることを望む

%%% Local Variables:
%%% mode: japanese-latex
%%% TeX-master: "../thesis"
%%% End:
