\chapter{結論}
\label{conclusion}

本章では第\ref{evaluation}章の3つの実験の結果に対する考察を行い、第\ref{introduction}章で述べた貢献\ref{kouken}にどのように繋がったか考察する。
そして、今後の問題点についてまとめ、将来的な展望について記す。


\section{本研究のまとめと解決した課題}
\label{matome}


 実験1(\ref{exp1}節)によりSigmoidでは精度が低いデータセットやReLUでは精度の低いデータセットでもK-AFはある程度高い精度が出せることが明らかになった。
これによりデータセットの形を意識せずとも同様の活性化関数を用いても問題なく学習を行えるので、初心者でも気軽に使える活性化関数になった。
 実験2(\ref{exp2}節)により状況に応じた活性化関数の形が実際可視化されることで、既存の活性化関数の有用性を示すだけでなく、
 新たな活性化関数の模索の必要性を示すことができた。
 また、分類系の問題ではSigmoidのような活性化関数でも表現の幅として程度十分であることも可視化され理解することができた。
これらの結果により、ブラックボックス化された活性化関数選択問題の解決に近づいた。
 実験3(\ref{exp3}節)ではこのようなK-AFがどのような場合に勾配爆発の少ない有効な活性化関数になるか定量的に示すことができた。

以上により\ref{honkadai}で述べた二つの課題について部分的に貢献し、\ref{kouken}節で述べた課題を解決した。



\section{本研究の問題点}
本研究をよりディープラーニング等のより実用的な問題に応用するためには以下の二つの課題を解決する必要がある。

\subsection{勾配の発散問題}
K-AFは一部のデータセットにおいては適切にInitializerを設定しなければ、勾配が爆発し学習が収束しない問題が本論文でも挙げられた。
今後はそのようなInitializerがどのようなものであるが研究を積み重ね、より良い確率で収束させることができるアルゴリズムを発見していく必要がある。

\subsection{ディープラーニングへの応用}
本論文ではK-AFは出力層の活性化関数のみを置き換え精度を上げることに成功した。
しかしながら、K-AFでは、次の層の正解のラベルデータを必要とするため、中間層に生かすことができない。
より良い精度を出すためには中間層でも使える新しいアルゴリズムが求められる。



\section{将来的な展望}

活性化関数を汎用的に推論するという論文は未だ少なく研究分野として今後非常に注目すべきである。 
将来的には自動運転などの産業分野においても有用なモデルへの応用されることを望む。
また、K-AFが汎用的な活性化関数の代表として初学者や非エンジニアが扱いやすい道具として応用されることを望む。


%%% Local Variables:
%%% mode: japanese-latex
%%% TeX-master: "../thesis"
%%% End:
