\chapter{結論}
\label{conclusion}

カーネルを使った汎用的な関数でニューラルネットの最終層を置き換えることで、実際に制度の向上を図ることができた。

\section{本研究のまとめ}

\section{本研究の課題}
スカラー値が大きなデータセットにおいては、その推論の精度が低下するだけではなく、Nan値に陥ってしまうことがわかる。
これらを解消するために、適切なニューラルネットの構成をより一層研究するだけではなく、それらが起こる原因を探求する必要がある。



\section{将来的な展望}
最終層だけではなく、中間層も置き換える。

%%% Local Variables:
%%% mode: japanese-latex
%%% TeX-master: "../thesis"
%%% End:
