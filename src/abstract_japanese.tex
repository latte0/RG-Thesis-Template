卒業論文要旨 - 20xx年度 (令和xx年度)
\begin{center}
\begin{large}
\begin{tabular}{|M{0.97\linewidth}|}
    \hline
      \title \\
    \hline
\end{tabular}
\end{large}
\end{center}

~ \\



近年ディープラーニングでは、シグモイド関数やReLU関数などの活性化関数が一般的に用いられてきた。
それらの活性化関数は,特定の条件下では高い精度を得ることができるが、あらゆる状況において最適化どうかはあまり議論されていない。
一方,統計学の分野では,リンク関数が未知の場合には,カーネル関数を用いてノンパラメトリックに推定するという手法が推定されている。
そこで本論文は事前に関数の形を指定しないディープラーニング活性化関数を提案する。
さらに、実際のデータセットを用いて、従来の活性化関数と同等かそれ以上の精度で予測できることを提これが可能であることを示した。



~ \\
キーワード:\\
\underline{1. ディープラーニング},
\underline{2. 活性化関数},
\underline{3. ノンパラメトリック},
\underline{4. カーネル関数}
\begin{flushright}
\dept \\
\author
\end{flushright}
