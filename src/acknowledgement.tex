\chapter*{謝辞}
\addcontentsline{toc}{chapter}{謝辞}
\label{thanks}

本論文の執筆にあたり、ご指導頂いた慶應義塾大学環境情報学部教授村井純博士、同学部教
授中村修博士、同学部教授楠本博之博士、同学部准教授高汐一紀博士、同学部教授三次仁
博士、同学部准教授植原啓介博士、同学部准教授中澤仁博士、同学部準教授 Rodney D。
Van Meter III 博士、同学部教授武田圭史博士、同大学政策・メディア研究科特任准教授
鈴木茂哉博士、同大学政策・メディア研究科特任准教授佐藤 雅明博士、同大学 SFC 研究
所上席所員斉藤賢爾博士に感謝致します.

特に斉藤氏には重ねて感謝致します。研究活動を通して技術的視点、社会的視点等の
様々な視点から私の研究に対して助言を頂き、深い思考と学びを経験させて頂くことがで
きました。これらの経験は私の人生において人・学ぶ者として、素敵な財産として残りま
した。博士の指導なしには、卒業論文を執筆することは出来ませんでした。

徳田・村井・楠本・中村・高汐・バンミーター・植原・三次・中澤・武田合同研究プロ
ジェクトに所属している学部生、大学院生、卒業生の皆様に感謝致します。研究会に所属
する多くの方々が各々の分野・研究で奮闘している姿を見て学んだことが私の研究生活を
より充実したものとさせました。

異なる分野同士が触れ合い、学び合う環境に出会えたことを嬉しく感じます。
また、NECO 研究グループとして多くの意見・発想・知見を与えてくださった、慶應義
塾大学政策メディア・研究科 阿部涼介氏、卒業生 菅藤佑太氏、在校生 島津翔太氏、
宮本眺氏、渡辺聡紀氏、梶原留衣氏、渡辺聡紀氏、木内啓介氏、後藤悠太氏、倉重健氏、
九鬼嘉隆氏、相原航平氏、小島大季氏、吉開拓人氏、金城奈菜海氏、長田琉羽里氏、
崔仁珠氏、上倉隼氏、田崎和輝氏に感謝致します。

村井純研究室統計的機械学習グループの小林凌雅氏、政策メディア研究科博士後期課程 川本章大氏、同研究科修士課程 幅野莞佑氏、
環境情報学部学士課程 岩崎智也氏、株式会 社 SIMULA Labs CEO 牧野暉弘氏、EMC Healthcare 株式会 社 深澤風土氏、
一橋大学経済 学研究科 田柳敏和氏に感謝します。
SL で共に研究について議論をし、学ぶことができた 経験は一生の財産になりました。重ねて感謝申し上げます。

特に研究について幾度となく議論に付き合ってくださった小林凌雅氏には重ねて感謝します。
研究のアイディアの段階から様々な指摘やアドバイスをしていただきました。
小林凌雅氏の存在なしにはこの論文は存在しなかったと思います。
研究の他にもここでは書ききれないほど多くのことで支えていただきました。
最後にもう一度重ねて感謝申し上げます。

皆様には、私の研究に対する多くの助言や発想を頂いただけでなく、研究活動における
学びを経験させて頂きました。
多くの出会いと学びの環境である SFC に感謝致します。多様な学問領域に触れ、学生
同士で議論し思考することが出来ました。幸せで素敵な時間でした。

最後に、これまで私を育て、見守り、多くの学びの機会並びに本卒業論文執筆の機会を
与えてくれた、両親と兄に感謝致します。



%%% Local Variables:
%%% mode: japanese-latex
%%% TeX-master: "../yummy_bthesis"
%%% End:
