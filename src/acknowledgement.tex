\chapter*{謝辞}
\addcontentsline{toc}{chapter}{謝辞}
\label{thanks}

本論文の執筆にあたり,ご指導頂いた慶應義塾大学環境情報学部村井純博士,同学部教
授中村修博士,同学部教授楠本博之博士,同学部准教授高汐一紀博士,同学部教授三次仁
博士,同学部准教授植原啓介博士,同学部准教授中澤仁博士,同学部準教授 Rodney D.
Van Meter III 博士,同学部教授武田圭史博士,同大学政策・メディア研究科特任准教授
鈴木茂哉博士,同大学政策・メディア研究科特任准教授佐藤 雅明博士,同大学 SFC 研究
所上席所員斉藤賢爾博士に感謝致します.
特に斉藤氏には重ねて感謝致します.研究活動を通して技術的視点,社会的視点等の
様々な視点から私の研究に対して助言を頂き,深い思考と学びを経験させて頂くことがで
きました.これらの経験は私の人生において人・学ぶ者として、素敵な財産として残りま
した.博士の指導なしには、卒業論文を執筆することは出来ませんでした.
徳田・村井・楠本・中村・高汐・バンミーター・植原・三次・中澤・武田合同研究プロ
ジェクトに所属している学部生,大学院生,卒業生の皆様に感謝致します.研究会に所属
する多くの方々が各々の分野・研究で奮闘している姿を見て学んだことが私の研究生活を
より充実したものとさせました.
異なる分野同士が触れ合い,学び合う環境に出会えたことを嬉しく感じます.
また,NECO 研究グループとして多くの意見・発想・知見を与えてくださった,慶應義
塾大学政策メディア・研究科 阿部涼介氏,卒業生 菅藤佑太氏,在校生 島津翔太氏,宮本
眺氏,松本三月氏,梶原留衣氏,渡辺聡紀氏,木内啓介氏,後藤悠太氏,倉重健氏,九鬼
嘉隆氏,内田渓太氏,山本哲平氏,吉開拓人氏,金城奈菜海氏,長田琉羽里氏,前田大輔
氏に感謝致します.
皆様には,私の研究に対する多くの助言や発想を頂いただけでなく,研究活動における
学びを経験させて頂きました.
多くの出会いと学びの環境である SFC に感謝致します.多様な学問領域に触れ,学生
同士で議論し思考することが出来ました.幸せで素敵な時間でした.
研究活動のおける実験協力をしていただいた皆様に感謝いたします.特に,製作資料を
提供して頂いた高島秀二郎氏,岡村奈於氏,栗原美優氏には,資料提供以外でも研究に関
して多くの助言を頂き,研究をより良いものとすることができました.
最後に,これまで私を育て,見守り,学びの機会を与えて頂いた,父 宏志氏,母 治子
氏,姉 遥香氏,姉 裕香氏に感謝致します.



%%% Local Variables:
%%% mode: japanese-latex
%%% TeX-master: "../yummy_bthesis"
%%% End:
