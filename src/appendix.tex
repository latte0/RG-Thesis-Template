\appendix
\chapter{ガウス分布とカーネル関数}


\section{カーネル活性化関数の導出}

以下の数式を出す

\begin{eqnarray}
G(X_iw)=\frac{\sum_{i\neq j} K\left(\frac{X^{calc}_j w - X_i w}{h_{calc}}\right)Y^{calc}_j}{\sum_{i\neq j} K\left(\frac{X^{calc}_j w - X_i w}{h_{calc}}\right)}
\label{eq:k-af-2}
\end{eqnarray}


\chapter{カーネル活性化関数の実装}


\section{クラス}


中間層が一つのkernel activation functionの計算を考慮した実装クラスを以下に示す。


\begin{lstlisting}[caption=hoge,label=fuga]
class Net(nn.Module):

    def __init__(self, Y, calc_Y, X, calc_X, settings):
        super(Net, self).__init__()

        self.fc1 = nn.Linear(DATA_INPUT_LENGTH, DATA_MID_LENGTH, bias=False)
        self.fc2 = nn.Linear(DATA_MID_LENGTH, DATA_OUTPUT_LENGTH, bias=False)
        # leave_ont_outのために事前に入力と出力をセットしておく
        self.Y = Y
        self.calc_Y = calc_Y
        self.calc = False
        # バンド幅も推定する
        self.h = nn.Parameter(torch.tensor(1.5, requires_grad=True))
        self.h_middle = torch.tensor(1.0)

        self.last_layer_result = []
        self.sigmoid = nn.Sigmoid()

    # kernel推定量の計算
    def kernel(self, Zw):
        numerator = 0
        denominator = 0
        result = []
        for j, x_j in enumerate(self.train_X):

            Xw = self.fc2(F.relu(self.fc1(x_j)))
            tmp = gauss((Xw - Zw) / self.h)

            tmp[j] = 0
            denominator += tmp
            numerator += tmp * self.Y[j]

        g = numerator/denominator
        return g


    def forward(self, x):

        xw = F.relu(self.fc1(x))
        xw = self.fc2(xw)

        y = self.kernel(xw)

        return y


\end{lstlisting}