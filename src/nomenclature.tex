\chapter*{記号}
\addcontentsline{toc}{chapter}{記号}
\label{thanks}


\begin{itemize}
    \item 1次元の値は通常の字体$ x $ を用い流。ベクトルや行列など、複数の値を内部に持っていることを強調したい場合は$ \mathrm{x} $ や $ \mathrm{X} $などの太字を用いる。
    \item $\mathbb{R}$ は実数の集合
    \item $\mathrm{E[x]}$ はxの期待値
    \item $e$ は自然対数の底で、指数関数は$\mathrm{exp}(x) = e^x$ のように表す
    \item $ \mathcal{G} $はガウス分布を表現する。
    \item 分布 $p(x)$からサンプルを得ることを$ x \sim p(x) $ と表記します。

\end{itemize}