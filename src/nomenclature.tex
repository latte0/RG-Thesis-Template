\chapter*{記号}
\addcontentsline{toc}{chapter}{記号}
\label{thanks}


\begin{itemize}
    \item 1次元の値は通常の字体$ x $ を用いる。ベクトルや行列など、複数の値を内部に持っていることを強調したい場合は$ \mathbf{x} $ や $ \mathbf{X} $などの太字を用いる。
    \item $\reals$ は実数の集合を表現し $\reals^d$は$ d $次元の実数のベクトル空間を表現する。 
    \item $\mathrm{E}[x]$ は$x$の期待値を表す。
    \item $e$ は自然対数の底で、指数関数は$\mathrm{exp}(x) = e^x$ のように表す
    \item $ \mathcal{G} $はガウス分布を表現する。
    \item 分布 $p(x)$からサンプルを得ることを$ x \sim p(x) $ と表記する。
    \item データセットの集合を$ \mathcal{D} $ で表現しその要素を $ d_i $と表現する。
    \item データセットの集合$ \mathcal{D} $ から$ n $個のデータをランダムにサンプリングすることを$ X \sim_n \mathcal{D}(X) $ 

\end{itemize}